%%%%%%%%%%%%%%%%%%%%%%%%%%%%%%%%%%%%%%%%%%%%%%%%%%%%%%%%%%%%%%%%%
% Contents: Who contributed to this Document
% $Id: overview.tex 456 2011-04-06 09:10:27Z oetiker $
%%%%%%%%%%%%%%%%%%%%%%%%%%%%%%%%%%%%%%%%%%%%%%%%%%%%%%%%%%%%%%%%%

% Because this introduction is the reader's first impression, I have
% edited very heavily to try to clarify and economize the language.
% I hope you do not mind! I always try to ask "is this word needed?"
% in my own writing but I don't want to impose my style on you...
% but here I think it may be more important than the rest of the book.
% --baron

\chapter{Preface}

Thank you for purchasing the \emph{MantaMate}!
The \emph{MantaMate} is a Eurorack module intended for interfacing a variety of
control devices with the world of Eurorack. As you may have guessed, the primary
device we had in mind was the \emph{Snyderphonics Manta}, but the module is in
no way limited to just the \emph{Manta}.

The \emph{MantaMate} combined with a traditional control device acts as a CV
converter.

The \emph{MantaMate} combined with the \emph{Manata} acts as a control device
as well as a fully-featured pitch and rhythm sequencer. These features include:

\begin{itemize}
  \item Two sequencers running in parallel, each of up to 32 steps
  \item Each sequencer can be a pitch or trigger sequencer
  \item Variable note length
  \item Variable CV control: four CV values per note, per sequencer allowing
  up to eight controllable CV outputs
  \item Pitch and CV glide
  \item Composition mode to chain together sequences into longer tracks
  \item Up to 89 (????) saved compositions, each of up to 32x2 sequences
  \item On-the-fly control for both in-studio and performance use
\end{itemize}

\noindent If you have any questions, feel free to reach out to \contrib{Jeff Snyder}{jeff@snyderphonics.com} \\


\endinput


%

% Local Variables:
% TeX-master: "lshort2e"
% mode: latex
% mode: flyspell
% End:
