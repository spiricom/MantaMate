%%%%%%%%%%%%%%%%%%%%%%%%%%%%%%%%%%%%%%%%%%%%%%%%%%%%%%%%%%%%%%%%%
% Contents: Nomenclature
% $Id: things.tex 536 2015-06-26 06:41:33Z oetiker $
%%%%%%%%%%%%%%%%%%%%%%%%%%%%%%%%%%%%%%%%%%%%%%%%%%%%%%%%%%%%%%%%%
\renewcommand{\chaptername}{Section}
\chapter{Nomenclature}
\begin{intro}
In order to avoid confusion due to ambigious terminology, this chapter
aims to explicitly define the language that will be used regarding
the \emph{MantaMate} as well as the pairing with the \emph{Manta} throughout.
\end{intro}

\section{Modular Terminology}

Considering the MantaMate exists in the Eurorack world, this manual assumes
knowledge of general modular synthesizer principals and will use terms associated
as one would expect.


\section{User-Interface Terminology}

This section defines the nomenclature that will be used to reference the
user interface of the MantaMate. In other words, every physical button
and input/output socket.

\subsection{MantaMate}
\begin{itemize}

  \item \texttt{IO} emph{MantaMate} \texttt{Input} will strictly refer to
  \item \texttt{Buttons} Saving? Loading? Presets????

\end{itemize}

\subsection{Manta}
\begin{itemize}

  \item \texttt{Hex} Blah blah blah hexes are hexagonal
  \item \texttt{Slider} Blah blah blah the sliders are actually
                  not tapered its just an optical illusion
  \item \texttt{Buttons} They do stuff!

\end{itemize}

\section{Abstract Terminology}
\begin{itemize}

  \item \texttt{Mode} Blah blah, trigger vs. pitch
  \item \texttt{Pattern} Blah blah blah patterns and presets differ
  \item \texttt{Preset} Blah blah blah presets
  \item \texttt{Composition} Blah blah blah still needs to be implemented
  \item \texttt{Preference} Blah blah blah, in the menu


\end{itemize}

%

% Local Variables:
% TeX-master: "lshort2e"
% mode: latex
% mode: flyspell
% End:
